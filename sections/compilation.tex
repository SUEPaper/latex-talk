\section{编译}

\begin{frame}[fragile]
  \frametitle{引擎与格式}
  \begin{itemize}
    \item<+-> \textbf{引擎}:\TeX{} 的实现

      \begin{itemize}
        \item \pdfTeX{}:直接生成 PDF,支持 micro-typography
        \item \XeTeX{}:支持 Unicode、OpenType 与复杂文字编排(CTL)
        \item \LuaTeX{}:支持 Unicode、OpenType,内联 Lua
        \item (u)p\TeX{}:日本方面推动,生成 |.dvi|,(支持 Unicode)
        \item Ap\TeX{}:底层 CJK 支持,内联 Ruby,Color Emoji(手动斜眼笑)
      \end{itemize}

    \item<+-> \textbf{格式}:\TeX{} 的语言扩展(命令封装)

      \begin{itemize}
        \item plain \TeX{}:Knuth 同志专用
        \item \LaTeX{}:排版科技类文章的事实\textit{de facto}标准
        \item Con\TeX t:基于 \LuaTeX{} 实现,优雅、易用(吗?)
      \end{itemize}

    \item<+-> \textbf{程序}:引擎 + dump 之后的格式代码

      \begin{itemize}
        \item \alert{英文文章:\pdfLaTeX{}、\XeLaTeX{} 或 \LuaLaTeX{}}
        \item \alert{中文文章:\XeLaTeX{} 或 \LuaLaTeX{}}
      \end{itemize}
  \end{itemize}
\end{frame}

\begin{frame}[fragile]
\frametitle{编译}
\begin{itemize}
\item 现代 \TeX{} 引擎均可直接生成 PDF \pause
\item 命令行

    \begin{itemize}
    \item |pdflatex|/|xelatex|/|lualatex| + |<文件名>[.tex]|
    \item 多次编译:读取并排版中间文件 \pause
    \item 推荐 \pkg{latexmk}:|latexmk [<选项>] <文件名>|

        \begin{itemize}
        \item |latexmk -xelatex main|
        \end{itemize}
    \end{itemize} \pause

\item 编辑器

    \begin{itemize}
    \item 按钮的背后仍然是命令
    \item |PATH| 环境变量:确定可执行文件的位置
    \item VS Code:配置 |tools| 和 |recipes|
    \end{itemize}
\end{itemize}
\end{frame}
