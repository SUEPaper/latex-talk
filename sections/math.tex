\section{数学公式}

\begin{frame}[fragile]{\LaTeX{} 数学模式}
    \begin{itemize}
    \item 数学公式排版是 \LaTeX{} 的绝对强项
    \item 一切数学公式都要在数学模式下输入,引用 \texttt{amsmath} 宏包,由美国数学学会(American Mathematical Society, AMS) 提供。
    \pause
    
    \item 
        \begin{itemize}
          \item 不受外界字体命令控制
          \item 数学模式中空格不起作用,尽管用;但不能有空行
          \item 建议始终调用 \pkg{amsmath} 宏包 \pause
          \item \alert{不建议用 MathType 生成 \LaTeX{} 公式}
          \item 但可以用 MathJax \link{https://www.mathjax.org} 或 KaTeX \link{https://katex.org} 练习
        \end{itemize}
    \end{itemize}
\end{frame}

\begin{frame}[fragile]
\frametitle{数学公式}
\begin{itemize}

  \item 行内(inline)公式

    \begin{itemize}
      \item 用一对美元符号(公式值千金):|$...$|
      \item 示例:理想气体状态方程可以写为 $PV=nRT$, 其中 $P$、$V$ 和 $T$
        分别是压强、体积和绝对温度
    \end{itemize} \pause

  \item 独显(display)公式

    \begin{itemize}
      \item 无编号:|\[...\]| 或 |equation*| 环境
      \item 编号:|equation| 环境
      \item \alert{不要用 \texttt{\$\$...\$\$}}
    \end{itemize}
\end{itemize}
\end{frame}

\begin{frame}[fragile]
\frametitle{结构}
\begin{itemize}
  \item<+-> 上下标

    \begin{itemize}
      \item |^| 和 |_|:|f^ab| 和 |f^{ab}|,|e^x^2|、|{e^x}^2| 和 |e^{x^2}|
      \item 张量:|R^a{}_b{}^{cd}| 或使用 \pkg{tensor} 宏包
      \item 配合积分、求和、极限使用:|\int|、|\sum|、|\lim|;
        \lstinline[style=style@inline]|\(no)limits|
    \end{itemize}

  \item<+-> 分式

    \begin{itemize}
      \item |\frac{〈分子〉}{〈分母〉}|
      \item 行内分式、小分式不好看:改用 |a/b|,或改用独显公式
      \item \alert{不推荐 \texttt{\textbackslash dfrac}}
    \end{itemize}

  \item<+-> 根式

    \begin{itemize}
      \item |\sqrt[〈次数〉]{〈内容〉}|
      \item 复杂情况改用分数指数:|{...}^{1/n}|
    \end{itemize}

  \item<+-> 矩阵与行列式

    \begin{itemize}
      \item |matrix|、|pmatrix|、|vmatrix| 等环境
      \item 语法类似表格:|&| 分列,|\\| 换行
      \item 推荐 \pkg{physics} 宏包
    \end{itemize}
\end{itemize}
\end{frame}

\begin{frame}[fragile]
\frametitle{括号与定界符}
\begin{itemize}
  \item<+-> 基本括号

    \begin{itemize}
      \item |(...)|、|[...]|、|\{...\}|、
      \item 绝对值、范数:\lstinline[style=style@inline]+|...|+ 或 |\vert...\vert|、|\Vert...\Vert|
      \item Dirac 符号:|\langle...\rangle|、\lstinline[style=style@inline]+|...\rangle+
    \end{itemize}

  \item<+-> 自动调节

    \begin{itemize}
      \item |\left(...\right)| 等
      \item 大型括号是拼出来的
    \end{itemize}

  \item<+-> 手动调节

    \begin{itemize}
      \item 只有 4 + 1 档:|\big|、|\Big|、|\bigg|、|\Bigg|
      \item 声明左中右:|\bigl|、|\bigm|、|\bigr| 等
    \end{itemize}
\end{itemize}
\end{frame}

\begin{frame}[fragile]
\frametitle{符号与字体}
\begin{itemize}
  \item 符号不是按钮点出来的,也不是天上掉下来的 \pause

    \begin{itemize}
      \item (几乎)所有的符号都由字体提供 \pause
      \item 分清「它是什么」和「它长什么样」(术语:character 和 glyph)
    \end{itemize} \pause

  \item 寻找符号

    \begin{itemize}
      \item 最常用的额外字体包:\pkg{amssymb}
      \item \LaTeX{} 公式大全 \link{https://suepaper.github.io/math201/docs/latex/math}
      \item 在线\LaTeX{}公式编辑器(支持图片识别) \link{https://www.latexlive.com/home}
    \end{itemize} \pause

  \item 数学字体

    \begin{itemize}
      \item 你们要的「Times New Roman」:\pkg{newtxmath} 宏包
      \item \alert{不要用 \pkg{times} 和 \pkg{mathptmx} 宏包}
      \item 加粗:使用 \pkg{bm} 宏包的 |\bm| 命令(|\mathbf| 只有直立的字母)
    \end{itemize} \pause

  \item 新方案:\pkg{unicode-math}

    \begin{itemize}
      \item 符号、字体、样式精调的一揽子解决方案
      \item 彻底修改底层,不可与传统方案混用
    \end{itemize}
\end{itemize}
\end{frame}

\begin{frame}[fragile]
\frametitle{多行公式}
\begin{itemize}
  \item 以下均要求 \pkg{amsmath} 宏包
  \item 独立数学环境

  \begin{itemize}
    \item 多行居中 |gather|、多行手动对齐 |align|、跨行 |multiline|
    \item 手动对齐:关系符前加 |&|
  \end{itemize}

  \item 内联数学环境

  \begin{itemize}
    \item 条件 |cases|、多行对齐 |split|、|...ed|
  \end{itemize} \pause

  \item 精细调整

  \begin{itemize}
    \item \pkg{mathtools}、\pkg{empheq} 等
    \item 自动换行:\pkg{breqn}
    \item \alert{避免使用 \texttt{eqnarray} 环境}
  \end{itemize}
\end{itemize}
\end{frame}



\begin{frame}[fragile]
    \frametitle{小露身手}

\begin{columns}
\begin{column}{.5\textwidth}
\lstset{language=[LaTeX]TeX}
\begin{lstlisting}[basicstyle=\ttfamily\small]
$V = \frac{4}{3}\pi r^3$

\[
    V = \frac{4}{3}\pi r^3
\]

\begin{equation}
\label{eq:vsphere}
V = \frac{4}{3}\pi r^3
\end{equation}
\end{lstlisting}
\end{column}

\begin{column}{.5\textwidth}
    $V = \frac{4}{3}\pi r^3$

    \[
        V = \frac{4}{3}\pi r^3
    \]
\begin{equation}
\label{eq:vsphere}
    V = \frac{4}{3}\pi r^3
\end{equation}
\end{column}
\end{columns}

\end{frame}


% \begin{frame}[fragile]
% \frametitle{小露身手}
%   \begin{equation*}
%     \oint \mathscr{D}[x(t)] \sqrt{\frac{3 \pi^{2}-\sum_{q=0}^{\infty}(z+\hat{L})^{q} \exp \left(\mathrm{i}^{2} \hbar x\right)}{(\operatorname{Tr} \mathscr{A})\left(\Lambda_{j_{1} j_{2}}^{i_{1} i_{2}} \Gamma_{i_{1} i_{2}}^{j_{1} j_{2}} \hookrightarrow \vec{D} \cdot \mathrm{P}\right)}}=
%     \underbrace{\widetilde{\left\langle \frac{\notin \emptyset}
%     {\varpi\alpha_{k\uparrow}}\middle\vert
%     \frac{\partial_\mu T_{\mu\nu}}{2}\right\rangle}}_{\mathrm{K}_3
%     \mathrm{Fe}(\mathrm{CN})_6} ,\forall z \in \mathbb{R}
%   \end{equation*}

%   \begin{lstlisting}[basicstyle=\ttfamily\small]
%     \begin{equation} % \usepackage{unicode-math}
%         \oint \mathscr{D}[x(t)] \sqrt{\frac{3 \pi^{2}-\sum_{q=0}^{\infty}(z+\hat{L})^{q} \exp \left(\mathrm{i}^{2} \hbar x\right)}{(\operatorname{Tr} \mathscr{A})\left(\Lambda_{j_{1} j_{2}}^{i_{1} i_{2}} \Gamma_{i_{1} i_{2}}^{j_{1} j_{2}} \hookrightarrow \vec{D} \cdot \mathrm{P}\right)}}=
%         \underbrace{\widetilde{\left\langle \frac{\notin \emptyset}
%         {\varpi\alpha_{k\uparrow}}\middle\vert
%         \frac{\partial_\mu T_{\mu\nu}}{2}\right\rangle}}_{\mathrm{K}_3
%         \mathrm{Fe}(\mathrm{CN})_6} ,\forall z \in \mathbb{R}
%     \end{equation}
%     \end{lstlisting}
% \end{frame}

