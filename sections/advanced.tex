\section{进阶扩展}

% \begin{frame}[fragile]
% \frametitle[Markdown]{%
%   Markdown \link{https://liam.page/2020/03/30/writing-manuscript-in-Markdown-and-typesetting-with-LaTeX}}
% \begin{columns}
% \lstset{%
%   moredelim    = [s][emphstyle]{*}{*},
%   moredelim    = [s][keywordstyle]{**}{**},
%   moredelim    = [s][emphstyle2]{\`}{\`},
%   moredelim    = [s][emphstyle2]{\`\`\`}{\`\`\`},
%   moredelim    = [l][keywordstyle2]{\#},
%   moredelim    = [is][keywordstyle2]{+}{+},
%   moredelim    = *[is][\itshape]{!}{!},
%   moredelim    = [is][keywordstyle]{(+}{+)},
%   moredelim    = [is][emphstyle2]{(-}{-)},
%   basicstyle   = \scriptsize\ttfamily,
%   keywordstyle = [1]\bfseries\color{keyword},
%   keywordstyle = [2]\bfseries\color{texcs},
%   emphstyle    = [1]\itshape\color{emph1},
%   emphstyle    = [2]\color{inline}}
% \begin{column}{0.48\textwidth}
%   \begin{lstlisting}[gobble=2]
%   # Markdown syntax

%   This is **bold text**.
%   This text is *italicized*.
%   Use `git status` to list all
%   new or modified files.

%   Block code:

%   ```
%   git status
%   git add
%   git commit
%   ```

%   Quotation:

%   +>+ !Markdown uses email-style `>`!
%   +>+ !characters for blockquoting.!
%   \end{lstlisting}
% \end{column}
% \begin{column}{0.48\textwidth}
%   \begin{lstlisting}[gobble=2]
%   ## List

%   ### Bullet list

%   +*+ apples
%   +*+ oranges
%   +*+ pears

%   ### Numbered list

%   +1.+ wash
%   +2.+ rinse
%   +3.+ repeat

%   +---+

%   Link: from [(+Wikipedia+)]
%   ((-https://en.wikipedia.org/wiki/-)
%   (-Markdown-))

%   \end{lstlisting}
% \end{column}
% \end{columns}
% \vspace{-0.6cm}
% \end{frame}

\begin{frame}[fragile]
\frametitle{Git}
\begin{itemize}
  \item<+-> 版本管理的必要性

    \begin{itemize}
      \item 远离「初稿,第二稿,第三稿……终稿,终稿(打死也不改了)」
      \item 有底气做大范围修改、重构
      \item 方便与他人协同合作
    \end{itemize}

  \item<+-> 基本用法

    \begin{itemize}
      \item 把大象放进冰箱:|git init|、|git add|、|git commit|
      \item 时空穿梭:|git reset|、|git revert|
      \item 平行宇宙:|git branch|、|git checkout|、|git rebase|
      \item 推荐用 VS Code 等进行可视化操作
      \item 参考链接:\link{https://git-scm.com/book}
        \link{https://www.liaoxuefeng.com/wiki/0013739516305929606dd18361248578c67b8067c8c017b000}
    \end{itemize}

  \item<+-> GitHub \href{https://github.com}{\faGithub} \& more

    \begin{itemize}
      \item 远程 Git 仓库
      \item Clone \& fork
      \item Issues \& pull requests
      \item<+-> \alert{提醒:绑定 \texttt{.edu} 邮箱可以有更多优惠}
    \end{itemize}
\end{itemize}
\end{frame}