\section{中文写作}
\begin{frame}{有关中文写作}
\begin{itemize}
    \item 宏包 \pkg{xeCJK}
    \item 参考 \url{https://www.overleaf.com/learn/latex/chinese}
\end{itemize}
\end{frame}

\begin{frame}[fragile]{中文示例}
  
    \begin{itemize}
        \item 编辑 \texttt{hello.tex} (Windows 下不要用中文文件名,注意
        \LaTeX{} 对文件名大小写敏感)
        \lstset{language=[LaTeX]TeX}
        \begin{lstlisting}[basicstyle=\ttfamily]
\documentclass{ctexart} % 使用中文适配的 article 文档类
\usepackage{xeCJK}%如果要在一般的文档内使用中文,一般只需引入此包
\begin{document}
\TeX{}你好!
\end{document}
          \end{lstlisting}
        \begin{itemize}
          \item Windows 下缺省使用中易字体
          \item Linux、macOS 下需要注意字体(参见 \pkg{ctex} 文档)
        \end{itemize}
      \item 使用 \XeLaTeX{} 引擎编译,得到 PDF 文档
        \begin{center}
          \fbox{\textrm \TeX{}\songti 你好!}
        \end{center}
    \end{itemize}
\end{frame}
  

